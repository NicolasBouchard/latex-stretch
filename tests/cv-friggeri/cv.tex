\documentclass{cv-friggeri}  % Use 'print' in option if needed

%\couleur % Ne pas utiliser pour une impression

\renewcommand{\nom}{Bouchard}
\renewcommand{\prenom}{Nicolas}

\usepackage[french]{babel}

\begin{document}


\begin{aside} % In the aside, each new line forces a line break
\section{contact}
514-953-5544
\section{langues}
Français
Anglais
\section{programmation}
Python,
\LaTeX{}, HTML/PHP,
Java, C/C++,
Mathematica, Matlab,
Git, Github
\end{aside}


\section{éducation}


\begin{entrylist}
\entryetc
{2013-2014}
{}{Complément de formation}
{Université de Montréal}
{Complétion des cours obligatoires au DEC en sciences de la nature}
\entryetc
{2012-2014}
{Maîtrise}{en mathématiques pures}
{Université de Montréal}
{spécialisation en théorie des nombres}
\entryetc
{2009-2012}
{Baccalauréat}{en mathématiques}
{Université de Montréal}
{orientation en mathématiques pures et appliquées}
\entryetc
{2007-2009}
{Diplôme d'études collégiales}{en musique}
{Cégep de Saint-Laurent}
{piano classique}
\end{entrylist}


\section{formation}


\begin{entrylist}
\entry
{2013}
{}{Maniement des armes à feu à autorisation restreintes} % Cours canadien de sécurité dans le maniement des armes à feu + Cours canadien de sécurité dans le maniement des armes à feu à autorisation restreintes
{ASFC}
\entry
{2013}
{}{Secourisme de base et RCR}
{Ambulance Saint-Jean}
\entry
{}
{}{Permis de conduire} % Classe 5
{}
\end{entrylist}


\section{expérience}


\begin{entrylist}
\entryetc
{2013-2014}  % Août 2013 à mai 2014
{Ambassadeur}{}
{Université de Montréal}
{Pour le Service de l'admission et du recrutement, représenter l'Université de Montréal dans les différentes écoles secondaires et les foires à l'éducation en plus de répondre aux courriels de questions sur l'admission.}
\entryetc
{2013-2014}  % Août 2013 à mai 2014
{Coadministrateur}{}
{Université de Montréal}
{Pour le Département de mathématiques et de statistique (DMS), administrer, conjointement avec les autres membre de l'équipe, les laboratoires informatiques et offrir des services aux usagers.
}
\entryetc
{2012-2013}
{Auxiliaire d'enseignement}{}
{Université de Montréal}
{Présenter les séances de travaux pratiques et effectuer de la correction pour les cours:
\begin{itemize}
  \item MAT2600 Algèbre I (Automne 2012)
  \item MAT1500 Mathématiques discrètes (Hiver 2013)
\end{itemize}
}
\entryetc
{été 2012}
{Assistant technique}{}
{Univeristé de Montréal}
{Pour le compte du Laboratoire informatique des systèmes adaptatifs (LISA), développer le logiciel de calcul formel Theano.
% \begin{itemize}
% \item Coder différentes parties du logiciel Theano
% \item Formaliser et vérifier les différents calculs
% \item Tester le code
% \item Reviser le code écrit par les autres développeurs.
% \end{itemize}
}
%------------------------------------------------
\entryetc
{2012}
{Consultant}{Soutien à la conception d’activité}
{Collège de Bois-de-Boulogne}
{Pour le compte du Centre de développement de la relève scientifique et technologique (CDRST), développer des activités à saveur scientifique pour les jeunes du secondaire en camps d'été.
% \begin{itemize}
%   \item Rédiger des activités à caractère scientifique pour des jeunes de première secondaire
%   \item Planifier les activités et établir les démarches de résolutions de problèmes
%   \item Expliquer des phénomènes scientifiques variés
% \end{itemize}
}
%------------------------------------------------
\entryetc
{2008-2013} % Mai 2008 à décembre 2013
{Agent des services frontaliers}{}
{ASFC - courrier international} % ETC-Léo Blanchette
{L'Agence des services frontaliers du Canada (ASFC) assure la sécurité et la prospérité du Canada en gérant l'accès des personnes et des marchandises lorsqu'elles arrivent au Canada ou en sortent.
% \begin{itemize}
%   \item Trier, référer, dédouaner et cibler les envois internationaux
%   \item S’assurer que les marchandises importées respectent les lois et règlements du pays
%   \item Établir le montant des droits et taxes des marchandises visées par le Tarif des douanes
%   \item Trouver et saisir les marchandises interdites d’importation
%   \item Créer les dossiers de saisie et effectuer le traitement des marchandises prohibées
% \end{itemize}
}
\end{entrylist}

% \newpage%
% \begin{aside}% In the aside, each new line forces a line break
% \section{contact}
% 514-953-5544
% \section{langues}
% Français
% Anglais
% \section{programmation}
% Python,
% \LaTeX{}, HTML/PHP,
% Java, C/C++,
% Mathematica, Matlab,
% Git, Github
% \end{aside}

\section{bourses, prix, distinctions}
\begin{entrylist}
\entry
{2012}
{Bourse de recherche}{}
{Laboratoire de théorie des nombres}
\entryetc
{2011}
{Bourse d’été de premier cycle}{}
{Institut des sciences mathématiques}
{Bourse de recherche d'été remise sur la base du dossier scolaire.}
\entryetc
{2011}
{Mémoire de fin d'étude}{(MAT4000)}
{Université de Montréal}
{Cours offerts aux meilleurs étudiants de la promotion}
\entryetc
{2010}
{1\iere{} place \emph{ex aequo}}{}
{Université de Montréal}
{Concours de graphiques et d’animations mathématiques}
\end{entrylist}


\section{conférences}


\begin{entrylist}
\entry
{2013}  % Été
{Présentation sur l'informatiques au DMS}{}
{en tant que coadministrateur}
\entry
{2012-2013}  % Été
{Séminaires \LaTeX{} I et II}{}
{en tant que coadministrateur}
\entryetc
{2012-2013}
{Comment voir les mathématiques}{}
{Projet SEUR}
{Présenté à 6 reprises, cette présentation vulgarisait la géométrie différentielle pour les jeunes du secondaire.}
\entry
{2012}  % Été
{Calcul formel et programmation}{}
{CCÉM}
\entry
{2011}  % Hiver
{Infini, intuition et paradoxes}{}
{SUMM}
\entry
{2011}  % Été
{Nombres congruents, courbes et fonctions elliptiques}{}
{}%{Mémoire de fin d'étude}
\entry
{2011}  % Été
{Des fonctions qui reviennent souvent}{}
{Séminaire d'été}
\entry
{2011}  % Été
{Une harmonie algébrique}{}
{CCÉM}
\end{entrylist}


\section{implication sociale et bénévolat}


\begin{entrylist}
\entry
{2013}
{Membre organisateur}{du Congrès canadien des étudiants en mathématiques (CCÉM)}
{}
%{En charge des technologies, des communications et de la publicité.}
\entry
{2012-2013}
{Conférencier}{pour le projet de Sensibilisation aux études universitaires et à la recherche (projet SEUR)}
{}
\entry
{2011-2012}
{Organisateur}{du club mathématique de l'Université de Montréal}
{}
\entry
{2011}
{Organisateur}{du séminaire d'été en mathématiques}
{}
\entry
{2011-2013}
{Participant}{aux multiples congrès et séminaires sur les mathématiques}
{}
\end{entrylist}

\section{articles}

\begin{itemize}
  \item Lamblin, P., Bastien, F., Pascanu, R., Bergstra, J, Goodfellow, I. Bergeron, A., Bouchard, N., Warde-Farley, D., Bengio, Y., \textbf{Theano : new features and speed improvements}. Article soumis au Deep Learning and Unsupervised Feature Learning Workshop pour le Neural Information Processing Systems Conference 2012.
\end{itemize}


\end{document}
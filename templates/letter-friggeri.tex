%%%
%%% Extension
%%%
\usepackage[utf8]{inputenc}
\usepackage[francais]{babel}
\usepackage{lmodern}
\usepackage{xcolor}
\usepackage{tikz}
\usetikzlibrary{calc}
\usepackage[allbordercolors=lightgray, pdfborderstyle={/S/U/W .2}]{hyperref}

%\usepackage{showframe}  % Pour faire des tests

%%%
%%% Variable
%%%
\newcommand{\nom}{}
\newcommand{\prenom}{}
\newcommand{\adresse}{}
\newcommand{\telephone}{}
\newcommand{\courriel}{}

\newcommand{\destinataire}{}
\newcommand{\destination}{}

\newcommand{\lieu}{}  % Où la lettre est écrite
\newcommand{\jour}{\today}  % Quand la lettre est écrite
\newcommand{\ouverture}{Madame, Monsieur,}
\newcommand{\salutation}{Cordialement,}
\newcommand{\signature}{\prenom~\nom}

%%%
%%% Couleur
%%%
\definecolor{white}{RGB}{255,255,255}
\definecolor{darkgray}{HTML}{333333}
\definecolor{gray}{HTML}{4D4D4D}
\definecolor{lightgray}{HTML}{999999}
\definecolor{green}{HTML}{C2E15F}
\definecolor{orange}{HTML}{FDA333}
\definecolor{purple}{HTML}{D3A4F9}
\definecolor{red}{HTML}{FB4485}
\definecolor{blue}{HTML}{6CE0F1}

%%%
%%% Macros
%%%
\newcommand{\cloture}{
\medskip\par\noindent{\salutation}%
\begin{flushright}%
\parbox{7cm}{\signature}
\end{flushright}%
}
\newcommand{\insertexpediteur}{%
\prenom~\nom\medskip\\
\telephone\\
\courriel%
}

\newcommand{\insertdestinataire}{%
\destinataire\\
\destination
}

\newcommand{\insertdatelieu}{%
\lieu,~\jour
}

\newcommand{\insertspace}{}
\newcommand{\insertobjet}{}
\newcommand{\objet}[1]{%
\renewcommand{\insertobjet}{%
Objet: {#1}\\
}%
}

\newcommand{\insertreference}{}
\newcommand{\reference}[1]{%
\renewcommand{\insertreference}{%
Ref: {#1}\\
}%
}

\newcommand{\insertpiecejointe}{}
\newcommand{\piecejointe}[1]{%
\renewcommand{\insertpiecejointe}{%
Pièce jointe: {#1}\\
}%
}

%%% Canva
\newlength{\marge}
\newlength{\margehaut}  % espace entre le haut et la date/expediteur
\newlength{\margepliage}  % espace entre le haut du papier et le début de la lettre sur la première page
\newlength{\espacepliage}  % espace entre le haut du corps du texte et le début de la lettre
\newlength{\hauteurrectangle}  % hauteur des rectangles haut et bas
\setlength{\marge}{2.5cm}
\setlength{\margehaut}{1.5cm}
\setlength{\hauteurrectangle}{.7cm}
\setlength{\margepliage}{.33\paperheight}
\setlength{\espacepliage}{\margepliage}
\addtolength{\espacepliage}{-\margehaut}
\addtolength{\espacepliage}{-1.5\baselineskip}  % TODO Pourquoi je dois ajuster à la main cette espace

%%% Tikz
\newcommand{\tikzrectangle}{%
\begin{tikzpicture} [remember picture, overlay]
  \node [rectangle, fill=gray, anchor=north, minimum width=\paperwidth, minimum height=\hauteurrectangle] (box-top) at (current page.north){};
  \node [rectangle, fill=gray, anchor=south, minimum width=\paperwidth, minimum height=\hauteurrectangle] (box-bottom) at (current page.south){};
  \node [rectangle, fill=blue, anchor=north west, minimum width=\hauteurrectangle, minimum height=\hauteurrectangle] at (current page.north west){};
  \node [rectangle, fill=red, anchor=north west, minimum width=\hauteurrectangle, minimum height=\hauteurrectangle] at ($(current page.north west) + (\hauteurrectangle, 0)$){};
  \node [rectangle, fill=orange, anchor=north west, minimum width=\hauteurrectangle, minimum height=\hauteurrectangle] at ($(current page.north west) + (2\hauteurrectangle, 0)$){};
%  \coordinate (numeropage) at ($(current page.south east) + (-\marge, .1)$);
%  \draw (numeropage) node [anchor=south east, color=white] {\thepage};
\end{tikzpicture}%
}

\newcommand{\tikzpliage}{%
\begin{tikzpicture} [remember picture, overlay]
  \coordinate (pliage) at ($(current page.north west) + (0, -\margepliage)$);
  \path [draw, dashed, color=lightgray] (pliage) -- ($(pliage) + (3\hauteurrectangle, 0)$);
\end{tikzpicture}%
}

\newcommand{\tikzexpediteur}{%
\begin{tikzpicture} [remember picture, overlay]
  \coordinate (expediteur) at ($(current page.north west) + (\marge, -\margehaut)$);
  \draw (expediteur) node [anchor=north west] {\parbox{6cm}{\insertexpediteur}};
\end{tikzpicture}%
}

\newcommand{\tikzdestinataire}{%
\begin{tikzpicture} [remember picture, overlay]
  \coordinate (destinataire) at ($(current page.north west) + (\marge, -.5\margepliage)$);
  \draw (destinataire) node [anchor=north west] {\parbox{10cm}{\insertdestinataire}};
\end{tikzpicture}%
}

\newcommand{\tikzdatelieu}{%
\begin{tikzpicture} [remember picture, overlay]
  \coordinate (datelieu) at ($(current page.north east) + (-\marge, -\margehaut)$);
  \draw (datelieu) node [anchor=north east] {\parbox{8cm}{\normalsize\raggedleft\insertdatelieu}};
\end{tikzpicture}%
}

%%%
%%% Style
%%%
\usepackage[margin=\marge, top=\margehaut]{geometry}
\setlength{\parskip}{\medskipamount}%

\usepackage{avant}
\renewcommand*\familydefault{\sfdefault}
\usepackage[T1]{fontenc}

\pagestyle{empty}
\newcommand{\couleur}{
\usepackage[all]{background}
\SetBgContents{\tikzrectangle}
\SetBgPosition{current page.north west}
\SetBgOpacity{1.0}
\SetBgAngle{0.0}
\SetBgScale{1.0}
}

%%%
%%% Entête
%%%
\AtBeginDocument{%
%\tikzrectangle%
\tikzpliage%
\tikzexpediteur%
\tikzdestinataire%
\tikzdatelieu%
\vskip\espacepliage%
\noindent{\insertreference%
\insertobjet%
\insertpiecejointe}%
\par\noindent{\ouverture}%
\vskip \medskipamount%
}



%%%
%%% Fin de la lettre
%%%
\AtEndDocument{}